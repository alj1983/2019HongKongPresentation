\documentclass[11pt]{scrartcl} 


\usepackage{graphicx,graphics,tikz,pgfkeys}
\usetikzlibrary{arrows,decorations.pathreplacing}
\usepackage{amsmath}
\usepackage{amsthm}
\usepackage{amsfonts}
\usepackage{amssymb}
\usepackage{gensymb}
\usepackage{textcomp}


\usepackage{fixltx2e} % to be able to use the command \textsubscript

\usepackage[amssymb,thinqspace,textstyle,binary,noams,derivedinbase,derived]{SIunits} % To use SI units.
% amssymb: This option redefines the amssymb command \square to get
% the desired SIunits definition of the command.  Note: When using
% this option, the amssymb command \square can not be used.

%thinqspace This mode provides the use of \, (thin math space) as spacing be-
%        tween numerical quantities and units.

% textstyle:  When using the option textstyle units are printed in the typeface of the
%enclosing text, automatically.

%binary   This option loads the file binary.sty, which defines prefixes for binary
%         multiples.
%noams This option redefines the \micro command; use it when you don’t have
%         the AMS font, eurm10.
%derivedinbase This mode provides the ready-to-use expressions of SI derived units
%         in SI base units, e. g. \pascalbase to get ‘m−1 kg s−2 ’.
%derived This mode provides the ready-to-use expressions of SI derived units in SI
%         derived units, e. g. \derpascal to get ‘N m−2 ’.


%\usetikzlibrary{arrows,decorations.pathmorphing,backgrounds,placments,fit}
\usepackage[graphics,tightpage,active]{preview}
\PreviewEnvironment{tikzpicture}
\newlength\imagewidth
\newlength\imagescale

\begin{document}

sdssss\pgfmathsetlength{\imagewidth}{20cm} % desired displayed width of image
\pgfmathsetlength{\imagescale}{\imagewidth/1200} % pixel width of image
% adjust scale of tikzpicture (and direction of y) such that pixel
% coordinates can be used for drawing overlays:


\begin{center}
\begin{tikzpicture}[font=\sffamily]

\node[inner sep=0pt,outer sep=0pt,scale=1,anchor=west] at (0cm,0cm) {\includegraphics[width=10cm]{{PiAbsHeatstress1.selected.without19.2AbolutePerformance}.png}};
\draw[color=red!60!black,line width=0.05cm] (2.1cm,1.6cm) rectangle (2.6,2.2);
\draw[color=red!60!black,line width=0.05cm] (2.1cm,-0.7cm) rectangle (2.6,0.15cm);

\draw[color=blue!60!black,line width=0.05cm] (6.55cm,-0.7cm) rectangle (7.05cm,1.25cm);
\draw[color=blue!60!black,line width=0.05cm] (6.55cm,-2.2cm) rectangle (7.05cm,-1.5cm);

\begin{scope}[xshift=2.5cm,yshift=-0.1cm]
\node[anchor=west,color=red!60!black,scale=0.9] at (-2cm,2.7cm) {Genes CG};
\node[anchor=east,color=red!60!black,scale=0.9] at (-0.4cm,2.2cm) {145 \textuparrow};
\node[anchor=east,color=red!60!black,scale=0.9] at (-0.4cm,1.7cm) {7 \textdownarrow};
\end{scope}
\node[anchor=west,color=blue!60!black,scale=0.9] at (5.2cm,2.6cm) {Intergenes/TEs CHG};
\node[anchor=west,color=blue!60!black,scale=0.9] at (7.1cm,1cm) {2 \textuparrow/4 \textuparrow};


% Genes CG
% 145
% 7

% Ig CHG
% 2

% TEs CHG
% 4

\end{tikzpicture}


\end{center}

\end{document}